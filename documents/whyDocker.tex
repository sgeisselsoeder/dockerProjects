\documentclass[a4paper, twoside, 11pt]{article}
% deutsche Silbentrennung
% \usepackage[ngerman]{babel}
% wegen deutschen Umlauten
\usepackage[utf8]{inputenc}

\usepackage{a4}

% fuer links im dokument
\usepackage[colorlinks, linkcolor = black, citecolor = black, filecolor = blue, urlcolor = blue]{hyperref}

% for code examples
\usepackage{listings}
% for equations with multiple lines
\usepackage{amsmath}

% for including graphics
\usepackage{graphicx}
% for floatbarriers
\usepackage{placeins}

\begin{document}

\title{Evaluation of lightweight virtualization using Docker in the context of ASTERICS to enhance reproducibility of analyses}
\author{Stefan Geißelsöder\\
ECAP, Erlangen}
\date{\today}
\maketitle

\section{Motivation}

As analyses in astronomy, astrophysics and particle astrophysics tend to increase in complexity 
with the abilities available on modern computer systems, 
the basic requirement of (short and long term) reproducibility is becoming harder to achieve. 
The high number of dependencies on other software packages, which have implicitly been used to obtain a result, 
are non-trivial to be reproduced exactly and sometimes not all dependencies are recognized explicitly. 

Techniques like Docker can help not only to achieve this reproducibility, 
but also in situations where checks or changes are desired after the know-how is not fully available anymore, 
e.g. when the original creator has left a collaboration, or when a new person is to become involved in a workflow. 

\section{Considerations that led to Docker}

Why do we recommend Docker instead of another technology for experiments in ASTERICS? 
Here are a few considerations of alternatives. 


\subsection{Plain scripts}

Plainly executing custom scripts on a machine in its currently available state 
risks not to be able to reproduce the computation and results with updated hard/software. 
This does not constitute an acceptable solution. 


\subsection{Virtual machines}

In principle virtual machines (VMs) are a reasonable idea to keep analyses reproducible. 
While there are different competing standards, most of the technologies are compatible with each other. 
Without additional care, VMs can decrease the efficiency of computations \cite{ieee:vmperformance}, but  
as the effect is not too big for our application 
(and can be reduced by certain VM techniques \cite{xen:about, ieee:vmperformance}), 
this is not considered a major drawback here. 
The main problem with VMs for the envisioned application is that the memory requirement doesn't 
scale good enough for a large number of analyses, 
since a whole virtual machine has to be stored for each analysis. 


\subsection{Docker}
Docker \cite{docker:about} is based on layered containerization. 
It allows direct access 
to the underlying kernel while abstracting the software dependencies. 
Since only the changes to a common base system required by an analysis have to be stored for an analysis, 
the main drawback of virtual machines in the envisioned application scenario is dealt with, 
namely the large memory requirements to store analyses. 
A drawback of the current status of Docker is the danger of ``privilege escalation'', 
which potentially allows users to perform actions on a machine they otherwise wouldn't have the rights to. 
This is no relevant concern for the application to physics analyses, 
but it prevents several clusters from supporting Docker directly. 
Desktop and cloud systems on the other hand support Docker. 


\subsection{Singularity}
Singularity \cite{singu:about} is another recent
option that could help with reproducibility similar to Docker. 
While it even deals with privilege escalations, 
so far it is not available on Microsoft Windows operating systems which are used by some members of some collaborations. 
Furthermore the already established user basis is not as large as Dockers 
and therefore the risk is higher that it might not be supported 
in the long run\footnote{Future developments can't be predicted, but a large existing user base tends to be 
correlated with a longer support and compatibility of following technologies.}. 


\subsection{Nix}
Nix \cite{nix:about} also constitutes a development that could be suited to help keeping analyses reproducible. 
However it focuses on an application and its dependencies. 
While it offers an elegant way to deal with these dependencies, it does not (and is not intended to) deal with e.g. input files. 
Furthermore it doesn't support Microsoft Windows and, 
judging from the current size of the Nix community, is also less widespread than Docker. 


\subsection{Others}
% flock none
There are many other, similar approaches 
(e.g. rkt \cite{rkt:about}, Flockport \cite{flockport:about}), 
but on the one hand, there is no essential feature lacking in Docker that would be required 
for our use-case. 
On the other hand, to achieve a long-term reproducibility, which is key to us, 
a solution should be supported by a large community and as many other cooperations or companies as possible. 
The current status is that this is fulfilled best by Docker. 
Furthermore, many alternatives also support Docker images or Dockerfiles, while their extensions usually are incompatible to others. 


\section{Summary}

Table \ref{tab:summary} is a summary of the comparison with perceived most relevant disadvantages in red.  
While it should be clear at this point that there is no single perfect solution, 
Docker seems to be the best choice. It does solve the problem of reproducibility of physics analyses, 
is relatively easy to use \cite{km3net:howdocker} 
and doesn't have major drawbacks as far as the investigated use-case is concerned. 

As this lightweight virtualization promises significant benefits, 
first efforts have already begun to employ Docker for KM3NeT \cite{github:dockerprojects}. 
The same concepts for packaging an analysis with Docker that are being worked out with KM3NeT 
(e.g. a common base image and a default input/output interface) can also be applied for other experiments.

\begin{center}
\begin{tabular}{ l | c | c | c | c | c }
\label{tab:summary}
      & Plain 	& VMs 	& Docker 	& Nix 	& Singularity \\
  \hline
  Reproducible & \textcolor{red}{Maybe not} 	& Yes 	& Yes		& Yes	& Yes \\
  \hline
  All OS & Yes 	& Yes 	& Yes		& \textcolor{red}{No}	& \textcolor{red}{No} \\
  \hline
  Widespread* & Yes & Yes & Yes & No & No \\
  \hline
  Efficient & Yes & Possible & Yes & Yes & Yes \\
  \hline
  Storage required & Low & \textcolor{red}{High} & Moderate & Low & Moderate \\
\end{tabular}
\end{center}
\bigskip
\begin{center}
*Community for different lightweight techniques on github.com, \\ state as of 2017-03-07 \\
\bigskip
\begin{tabular}{ l | c | c }
\label{tab:popularity}
      & Commits	& Contributors \\
  \hline
  Docker & 31219 & 1627 \\
  \hline
  Nix & 2048 & 34 \\
  \hline
  Singularity & 5075 & 110 \\
  \hline
  rkt & 5167 & 177 \\
  \hline
\end{tabular} \\
\end{center}

\bibliography{dockerRefs}
\bibliographystyle{alpha}

\end{document}
